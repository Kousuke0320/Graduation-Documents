% 公立はこだて未来大学 卒業論文 テンプレート ver1.50
% (c) Junichi Akita (akita@fun.ac.jp), 2003.10.31
% update by N.T.,  2004.11.10
%
\documentclass{funthesis}
%\documentclass[english]{funthesis} % use [english] option for English style

\usepackage{graphicx} % 図(EPS形式)を本文中で読み込む場合はこれを宣言

% この部分に,タイトル・氏名などを書く.
% タイトルなどの定義の始まり
\jtitle{日本語タイトル\\
--- サブタイトル ---
}  % 論文の和文タイトル
%
\etitle{Title in English\\
--- English Subtitle ---
}% 論文の英文タイトル
%
\htitle{Short Title in English}   % ヘッダー用の論文の短縮英文タイトル
%     必ず1行に収まるように英文タイトルを短縮する.
%
\jauthor{未来 太郎}     % 氏名(日本語)
\eauthor{Taro MIRAI}   % 氏名(英語)
\jaffiliciation{複雑系アーキテクチャ学科} % 所属学科名(日本語)
\eaffiliciation{Department of Complex Media Architecture} % 所属学科名(英語)
\studentnumber{1011000}   % 学籍番号
\jadvisor{正指導 教員}    % 正指導教員名(日本語)
\jcoadvisor{副指導 教員} % 副指導教員(日本語)がいる場合は
                        % コメントアウトし名前を書く
                        % 副指導教員がいない場合は,ここは削除しても可
\eadvisor{Prof. Advisor}  % 正指導教員名(英語)
\ecoadvisor{Prof. Coadvisor}   % 副指導教員(英語)がいる場合は
                         % コメントアウトし名前を書く
                         % 副指導教員がいない場合は,ここは削除しても可
\jdate{2014年1月31日}    % 論文提出日   (日本語)
\edate{January 31, 2014}     % 論文提出年月 (英語)
% タイトルなどの定義の終わり

\begin{document}

%--------------------------------------------------------------------
\maketitle       % タイトルページを作成

%--------------------------------------------------------------------
% 英文概要(250語程度)
\begin{eabstract}
Abstract in English. (about 250 words)
About Object Oriented Language.
one two three four five six seven eight nine ten
one two three four five six seven eight nine ten
one two three four five six seven eight nine ten
one two three four five six seven eight nine ten
one two three four five six seven eight nine ten
one two three four five six seven eight nine ten
one two three four five six seven eight nine ten
one two three four five six seven eight nine ten.
one two three four five six seven eight nine ten
one two three four five six seven eight nine ten
one two three four five six seven eight nine ten
one two three four five six seven eight nine ten
one two three four five six seven eight nine ten
one two three four five six seven eight nine ten
one two three four five six seven eight nine ten.
one two three four five six seven eight nine ten
one two three four five six seven eight nine ten
one two three four five six seven eight nine ten
one two three four five six seven eight nine ten
one two three four five six seven eight nine ten
one two three four five six seven eight nine ten
one two three four five six seven eight nine ten
one two three four five six seven eight nine ten
one two three four five six seven eight nine ten.
\end{eabstract}

% 英文キーワード(5個程度をコンマ(,)で区切って羅列する)
\begin{ekeyword}
Keyrods1, Keyword2, Keyword3, Keyword4, Keyword5
\end{ekeyword}

%--------------------------------------------------------------------
% 和文概要(400字程度)
\begin{jabstract}
日本語の概要を書く.(約400字,英文概要と合わせて0.8-1ページ程度)

オブジェクト指向言語の研究をおこなった.
(以下の内容は本文を含め,サンプルゆえ,
荒唐無稽なものとなっています.)

いろはにほへとちりぬるをわかよたれそつねならむういのおくやまけふこえて
あさきゆめみしえひもせす
いろはにほへとちりぬるをわかよたれそつねならむういのおくやまけふこえて
あさきゆめみしえひもせす
いろはにほへとちりぬるをわかよたれそつねならむういのおくやまけふこえて
あさきゆめみしえひもせす
いろはにほへとちりぬるをわかよたれそつねならむういのおくやまけふこえて
あさきゆめみしえひもせす

いろはにほへとちりぬるをわかよたれそつねならむういのおくやまけふこえて
あさきゆめみしえひもせす
いろはにほへとちりぬるをわかよたれそつねならむういのおくやまけふこえて
あさきゆめみしえひもせす
いろはにほへとちりぬるをわかよたれそつねならむういのおくやまけふこえて
あさきゆめみしえひもせす
いろはにほへとちりぬるをわかよたれそつねならむういのおくやまけふこえて
あさきゆめみしえひもせす

\end{jabstract}

% 和文キーワード(5個程度をコンマ(,)で区切って羅列する)
\begin{jkeyword}
キーワード1, キーワード2, キーワード3, キーワード4, キーワード5
\end{jkeyword}

%--------------------------------------------------------------------
\tableofcontents % 目次を作成


% 本文のはじまり
%--------------------------------------------------------------------
\chapter{序論} % 章のタイトル
%\chapter{Introduction} % sample of English style

123456789012345678901234567890
123456789012345678901234567890
123456789012345678901234567890
123456789012345678901234567890
123456789012345678901234567890

123456789012345678901234567890
123456789012345678901234567890
123456789012345678901234567890
123456789012345678901234567890
123456789012345678901234567890

% \includegraphics[width=??cm]{hoge.eps} % 図(EPS形式)を読み込む場合

\section{背景} % sectionのタイトル

% 以下に背景,関連する環境,状況,技術に関する概要を記述.

手続き型言語では,巨大システムを構築し,管理を行うことが難しいため,こ
こにオブジェクト指向という新たな考え方を導入して新しいプログラミング言
語を作成することにした.

\section{対象とする領域}

実用レベルのサイズのプログラムを作成するためのプログラミング言語につい
て研究する.ここで,行うのは3次元グラフィックス向けの言語の設計とその
インタプリタの実装である.

\section{研究目標}

完全な処理系の実装を目指すものではなく,プログラミング言語にオブジェク
ト指向という考え方を取り入れたプログラミング言語を設計し,プロトタイプ
システムを作成することにより,オブジェクト指向の概念が,プログラミング
の能率向上とメンテナンス性の向上に寄与することを示す.

%--------------------------------------------------------------------
\chapter{関連研究}

\section{オブジェクト指向プログラミング}

\subsection{Smalltalk-80} % subsectionのタイトル

Smalltalk-80は1982年ごろ,当時ゼロックスにいた...

\subsubsection{必要があれば} % subsubsectionのタイトル
% ※ subsubsectionはあまり使わないほうがよい

\subsection{Java 3D}

Javaはオブジェクト指向言語で,そこで3D グラフィックスを扱うための..

\section{グラフィックスシステム}

\subsection{DirectX}

DirectX はマイクロソフトのWindows上の.....


%--------------------------------------------------------------------
\chapter{プログラミング言語FUN}

この章では,提案する理論,仮説,モデル,アルゴリズム,
方法論,実装のなどの説明を行う.

\section{提案する言語FUNの特徴}

この言語の特徴は,..であり,...という従来にない長所をもつ.

\section{言語仕様}

言語仕様は以下の通り.


\section{実装方法}

この言語は,C言語を用いて記述されている.ソースコードは20に分かれ,
コードの大きさは約3000行となった.

\subsection{開発環境}

この言語は,C言語を用いて記述されている.ソースコードは20に分かれ,
コードの大きさは約3000行となった.

\subsection{OSに対する依存性}

この言語は,C言語を用いて記述されている.ソースコードは20に分かれ,
コードの大きさは約3000行となった.


%--------------------------------------------------------------------
\chapter{実験と評価}

\section{保守性に関する評価}

ここでは,FUNを用いて記述した場合と
それ以外の言語で書いた場合の比較を行なう.

\subsection{Fortranとの比較}

同一のゲームをFortranとFUNで記述してみた.

\subsubsection{スーパーマリオブラザーズ}

一見,このプログラムはFortran向きと考えられるが,
FUNのTAKOIKAライブラリを用いて記述すると,
非常にコンパクトになる.

\subsubsection{パックマン}

このプログラムはどちらの言語にとっても,
有利な要素はない,このことを反映して.

\subsection{Javaとの比較}

Java言語との比較では,惨敗であり,FUNは2倍の
記述量を必要とした.しかし,これは,Javaのもつ
パッケージIKURAが非常に強力であるためで,
同一機能をもつライブラリを用意することにより,
FUNにも同様の能力を持たせることができることが判明した.

\section{実行速度}

\subsection{Fortranとの比較}

Java言語との比較では,惨敗であり,FUNは2倍の
記述量を必要とした.しかし,これは,Javaのもつ
パッケージIKURAが非常に強力であるためで,
同一機能をもつライブラリを用意することにより,
FUNにも同様の能力を持たせることができることが判明した.

\subsection{Javaとの比較}

Java言語との比較では,惨敗であり,FUNは2倍の
記述量を必要とした.しかし,これは,Javaのもつ
パッケージIKURAが非常に強力であるためで,
同一機能をもつライブラリを用意することにより,
FUNにも同様の能力を持たせることができることが判明した.

\section{利用者によるアンケート}

\subsection{初心者}

Java言語との比較では,惨敗であり,FUNは2倍の
記述量を必要とした.しかし,これは,Javaのもつ
パッケージIKURAが非常に強力であるためで,
同一機能をもつライブラリを用意することにより,
FUNにも同様の能力を持たせることができることが判明した.

\subsection{上級者}

Java言語との比較では,惨敗であり,FUNは2倍の
記述量を必要とした.しかし,これは,Javaのもつ
パッケージIKURAが非常に強力であるためで,
同一機能をもつライブラリを用意することにより,
FUNにも同様の能力を持たせることができることが判明した.


%--------------------------------------------------------------------
\chapter{考察}

\section{評価結果}

Java言語との比較では,惨敗であり,FUNは2倍の
記述量を必要とした.しかし,これは,Javaのもつ
パッケージIKURAが非常に強力であるためで,
同一機能をもつライブラリを用意することにより,
FUNにも同様の能力を持たせることができることが判明した.

\section{評価結果}

Java言語との比較では,惨敗であり,FUNは2倍の
記述量を必要とした.しかし,これは,Javaのもつ
パッケージIKURAが非常に強力であるためで,
同一機能をもつライブラリを用意することにより,
FUNにも同様の能力を持たせることができることが判明した.


%--------------------------------------------------------------------
\chapter{結論と今後の展開}

\section{まとめ}

Java言語との比較では,惨敗であり,FUNは2倍の
記述量を必要とした.しかし,これは,Javaのもつ
パッケージIKURAが非常に強力であるためで,
同一機能をもつライブラリを用意することにより,
FUNにも同様の能力を持たせることができることが判明した.

Java言語との比較では,惨敗であり,FUNは2倍の
記述量を必要とした.しかし,これは,Javaのもつ
パッケージIKURAが非常に強力であるためで,
同一機能をもつライブラリを用意することにより,
FUNにも同様の能力を持たせることができることが判明した.

Java言語との比較では,惨敗であり,FUNは2倍の
記述量を必要とした.しかし,これは,Javaのもつ
パッケージIKURAが非常に強力であるためで,
同一機能をもつライブラリを用意することにより,
FUNにも同様の能力を持たせることができることが判明した.

\section{今後の方針}

Java言語との比較では,惨敗であり,FUNは2倍の
記述量を必要とした.しかし,これは,Javaのもつ
パッケージIKURAが非常に強力であるためで,
同一機能をもつライブラリを用意することにより,
FUNにも同様の能力を持たせることができることが判明した.


%--------------------------------------------------------------------
\chapter*{謝辞}

本研究において、長期にわたる評価実験に協力いただきました、株式会社○○の△△△△様に感謝いたします.


%--------------------------------------------------------------------
% 参考文献
\begin{thebibliography}{9}
 \bibitem {A1} アイン・シュタイン, 「相対性理論について」, 2000.
\end{thebibliography}


% 以降,付録(付属資料)であることを示す
\appendix

%--------------------------------------------------------------------
\chapter*{付録その1} % \chapter{}を使うと「付録A ***」となる

付録その1(プログラムのソースリストなど)を必要があれば載せる

%--------------------------------------------------------------------
\chapter*{付録その2}

付録その2(関連資料など)を必要があれば載せる

%--------------------------------------------------------------------
% 図一覧
\listoffigures

%--------------------------------------------------------------------
% 表一覧
\listoftables

\end{document}
