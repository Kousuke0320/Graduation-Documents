% 公立はこだて未来大学 卒業論文 テンプレート ver1.50
% (c) Junichi Akita (akita@fun.ac.jp), 2003.10.31
% update by N.T.,  2004.11.10
%
\documentclass{funthesis}
%\documentclass[english]{funthesis} % use [english] option for English style

\usepackage[dvipdfmx]{graphicx} % 図(EPS形式)を本文中で読み込む場合はこれを宣言

% この部分に,タイトル・氏名などを書く.
% タイトルなどの定義の始まり
\jtitle{デザインプロセスにおけるグループワーク編成支援ツールの提案\\
--- サブタイトル ---
}  % 論文の和文タイトル
%
\etitle{Suggestion of group work support tool in the design process\\
--- English Subtitle ---
}% 論文の英文タイトル
%
\htitle{Short Title in English}   % ヘッダー用の論文の短縮英文タイトル
%     必ず1行に収まるように英文タイトルを短縮¥¥¥¥¥¥¥¥¥¥¥¥¥¥¥¥

%
\jauthor{田中 康介}     % 氏名(日本語)
\eauthor{Kosuke Tanaka}   % 氏名(英語)
\jaffiliciation{情報アーキテクチャ学科} % 所属学科名(日本語)
\eaffiliciation{Department of Media Architecture} % 所属学科名(英語)
\studentnumber{1015013}   % 学籍番号
\jadvisor{姜 南圭}    % 正指導教員名(日本語)
\eadvisor{Prof. Kang}  % 正指導教員名(英語)
\jdate{2019年1月29日}    % 論文提出日   (日本語)
\edate{January 29, 2019}     % 論文提出年月 (英語)
% タイトルなどの定義の終わり

\begin{document}

%--------------------------------------------------------------------
\maketitle       % タイトルページを作成

%--------------------------------------------------------------------
% 英文概要(250語程度)
\begin{eabstract}
Currently, problems that designers have to deal with are considering the relationship with the users and the society. Therefore, it becomes difficult to find the best answer with the sensitivity and creativity of individual designers.As a result, it is required to organize teams by members with various skills and to conduct creative activities systematically.In the previous studies, results are gotten by distributing human resources in a well-balanced manner.But, it has not been reported what kind of influence the member composition has on creative activities.Consequently, we conducted a survey about group work in creative activities.As a result, most people chose group work rather than personal work for different reasons such as expanding their own vision.In this research, we suggest a web application, named ‘Skill Pentagon’ for organizing more various groups by visualizing the user's skill.\end{eabstract}

% 英文キーワード(5個程度をコンマ(,)で区切って羅列する)
\begin{ekeyword}
Design Process, Group work, Web application, Creativity
\end{ekeyword}

%--------------------------------------------------------------------
% 和文概要(400字程度)
\begin{jabstract}
現在デザイナーが対処すべき問題は,ユーザや社会との関係性をより丹念に考慮することが 求められるようになり,デザイナー個人の感性や創造性だけでは最適な答えを出すことは難しくなっている.このことから,多種多様なスキルを持つメンバーによってチームを組み,組織的に創造活動をすることが求められている.  先行研究では,人的リソースをバランスよく配分することで全体的な成果を挙げたことが報告されている.しかし,個人の創造活動がメンバー構成によってどのように変化したかは報告されていない.そのため,公立はこだて未来大学にて創造活動におけるグ ループワークの現状についてのアンケート調査を行った.その結果,視野が広がるなどの理由から,自分と異なるスキルを持つ人とグループワークを支持する人が大多数であるということが分かった. 本稿ではユーザのスキルを可視化することにより,より多種多様なグループ編成を行うための Web アプリケーションの提案を行う.
\end{jabstract}

% 和文キーワード(5個程度をコンマ(,)で区切って羅列する)
\begin{jkeyword}
デザインプロセス, グループワーク,ウェブアプリケーション, 創造性
\end{jkeyword}

%--------------------------------------------------------------------
\tableofcontents % 目次を作成


% 本文のはじまり
%--------------------------------------------------------------------
\chapter{序論} % 章のタイトル
%\chapter{Introduction} % sample of English style

本章では、本研究における背景と、研究目的について述べる.

% \includegraphics[width=??cm]{hoge.eps} % 図(EPS形式)を読み込む場合

\section{背景} % sectionのタイトル

% 以下に背景,関連する環境,状況,技術に関する概要を記述.

現在デザイナーが対処すべき問題は,ユーザや社会との関係性をより丹念に考慮することが求められるようになり,デザイナー個人の感性や創造性だけでは最適な答えを出すことは難しくなっている.そこで個人の創造性を超えて,多角的な方面からより創造的な解を生み出すために多様な専門性を持つチームによる組織的なデザイン行為のあり方について議論されることが増えてきた.\cite{A1} 石井らは創造性という観点から一般的な創造活動の認知モデルであるジェネプロアモデルの枠組みを適応し,問題解決においてアイデアを検討する段階に限定したうえで,「独立して考える場合と比較して二人で話し合うという協同には効果がある」と示した\cite{A2}. また,Kangは異なる学問や経験,文化を持っている人が集まりグループワークを行うことは,大きな創造性を生み出す潜在能力を持っている\cite{A3}としている.\\
\ これらのことから,自分と異なるスキルや視点を持つ人と共に創造的な作業を行うことにより,自分だけの潜在能力では考えることができなかった成果をあげることがあると考えられる.\\
\ しかし、グループワークでは, グループを形成するメンバーによっては、手を抜く学生が生じ、一人に過剰な負荷がかかるなどの問題も指摘されていることもあり\cite{A4}様々なグループ形成方法が議論されているのが現状である.

\section{研究目的}
本研究では創造性とグループワークに着目し,  個人のスキル,情報を可視化し,グループワークをする際により多種多様なスキルを持つメンバーによってグループを構成するためのツールの提案を行う.
%--------------------------------------------------------------------
\chapter{関連研究}

\section{創造性とグループワーク}

\subsection{国際デザインワークショップ} % subsectionのタイトル
国際デザインワークショップがががががががが
\subsection{Java 3D}

Javaはオブジェクト指向言語で,そこで3D グラフィックスを扱うための..

\section{グループ編成}

\subsection{DTMPメソッドを用いたグループ編成システム}

相島・塚原・植木.杉浦\cite{A5}は学習プロセス方法論であるDTMPメソッドをユーザのふるまいに当てはめ,人的リソースを配分する手法を提案し,ウェブアプリケーションを実装,グループ学習において検証を行った.DTMPメソッドとは,  DTMPの各系統の能力について以下の図のように定義したものである表\ref{DTMP}.
\begin{table}[h]
\begin{center}
  \begin{tabular}{cll} \hline
    D & Design & 創造的な本質を理解し,創造活動を実践できる能力\tabularnewline
    T & Technology & 
    創造的活動を支えるデジタルメディア関連の技術を理解・活用できる能力\tabularnewline
    M &Management & 
    創造的デザインプロセスを管理できる能力\tabularnewline
    P &Polisy &
    創造的活動の成果を戦略的に活用でき,創造的プロセスを取り巻く政策を理解できる能力\tabularnewline
    \hline
  \end{tabular}
  \caption{DTMPメソッドの定義}
  \label{DTMP}
  \end{center}
\end{table}

\ ふるまいはユーザの潜在的な特性を反映しており,DTMP系統の特性を対応づけることにより,個人が持つDTMP特性を抽出できる.\\
\ 抽出した各DTMP特性を元にメンバーを均等に分配し,2ヶ月間のグループワークを行なった.
\ 検証を行った結果,人的リソースのバランス良い配分という観点から効果を認めることができたが,DTMPメソッドの各特性が持つ能力や傾向を確認することはできなかった.

\subsection{ソフトウェア開発グループ演習のためのチーム最適化支援}
橋浦・桑原・秋・石川・山下・古宮\cite{A6}はソフトウェア開発におけるグループ内の役割をリーダー,設計担当,コーディング担当,品質管理担当に分け,それぞれに必要な適正を成績から代用し,遺伝的アルゴリズムを使用してチーム編成を行う手法を提案した.
\begin{table}[h]
\begin{center}
  \begin{tabular}{lll} \hline
    役割(Y) & 適正 & 代用適正の例\tabularnewline
    リーダ & PM能力 & プロジェクトマネジメントに関する問題の成績\tabularnewline
    設計担当 &分析・設計能力 & 
    ソフトウェア設計に関する問題の成績\tabularnewline
    コーディング担当&プログラミング能力 &
    プログラミングに関する問題の成績\tabularnewline
    品質保証担当&品質管理能力 &
    ソフトウェアテスト技術に関する問題の成績\tabularnewline
    \hline
  \end{tabular}
  \caption{役割 - 適正 - 代用適正の関係(例)}
  \label{適正}
  \end{center}
\end{table}
ソフトウェア開発グループ演習において検証を行った結果,学習者の満足度が上がったことと,チーム間の成果物のバラつきが軽減されたことから,ソフトウェア開発演習のための最適なチーム編成が実現できていることを確認した.


\subsection{学習者の思考特性に着目したグループ形成支援の方法}
井上・埴生は\cite{A7}組織最適編成法理論のひとつであるFFS理論(FIve Factors and Stress Theory)
\cite{A8}を用いて,学習者の特性,すなわち,人の個性や潜在能力傾向に着目した協調学習のためのグループ形成方法についての検討を行なった.
\ FFS理論とは,A.凝縮性,B.受容性,C.弁別性,D.拡散性,E.保全性の5つの因子とストレス値をチェックリストによって計量し,A,B,C,D,E因子の計量値から4種類のパーソナリティタイプ(リーダーシップ,マネジメント,タグボート,アンカー)に分類し、各因子がストレス状態によって表出される特徴を表\ref{FFS}のように示したものである.\cite{A9}
\begin{figure}[h]
 \centering
   \includegraphics[width=100mm]{figures/FFS.png}
 \caption{パーソナリティタイプの分類}
 \label{FFS}
\end{figure}

\begin{table}[h]
\begin{center}
  \begin{tabular}{lll} \hline
    状態 & ポジティブ反応 & ネガティブ反応\tabularnewline
    A 凝縮性& 規範的,指導的 & 
    独善的,支配的\tabularnewline
    B 受容性 &肯定的,養育的 & 
    介入的,自虐的\tabularnewline
    C 弁別性&分析的,論理的 &
    機械的,詭弁的\tabularnewline
    D 拡散性&創造的,活動的 &
    衝動的,破壊的\tabularnewline
    E 保全性&順応的,協調的 &
    追随的,妥協的\tabularnewline
    \hline
  \end{tabular}
  \caption{各因子のストレス状態における特徴}
  \label{FFS理論}
  \end{center}
\end{table}

\begin{table}[h]
\begin{center}
  \begin{tabular}{lll} \hline
    Personality Type &  コンピテンシー\tabularnewline
   Leadership(LD)& 組織の先頭に立って変革を進める力  \tabularnewline
    Management(MG) & 取り巻く環境を判断しながら継続的に改善する力 \tabularnewline
    Tugboat(TG)& 環境の変化を捉えリスクに挑戦していく力\tabularnewline
    Anchor(AN)&リスクヘッジしながら手続き通りに着実に実行する力 \tabularnewline
    \hline
  \end{tabular}
  \caption{パーソナリティタイプとコンピテンシー}
  \label{FFS理論2}
  \end{center}
\end{table}

FFS理論で各因子を抽出したうえで,グループでの協調学習場面を想定し,創造性,効率性を対立軸として設け,4人のチーム編成を行なった.表\ref{FFS理論3}

\begin{table}[h]
\begin{center}
  \begin{tabular}{ll} \hline
    創造性を重視したグループ編成\tabularnewline
   1. 補完型グループ& 異質な型のメンバーで構成  \tabularnewline
   2. 意見発散型グループ & LM,AN,TG,TGタイプのメンバーで構成 \tabularnewline
    3. アイデア型グループ& 全員TGタイプのメンバーで構成\tabularnewline
    効率性を重視したグループ編成 \tabularnewline
    4. 同質型グループ& TGタイプを除く同質タイプのみのメンバーで構成  \tabularnewline
    5. 意見収束型グループ & LM,ML,ANタイプとTGタイプ以外のメンバーで構成 \tabularnewline
    6. リーダー主導型グループ&LMタイプが一人とLMタイプ以外の同質メンバーで構成\tabularnewline
    \hline
  \end{tabular}
  \caption{グループ編成}
  \label{FFS理論3}
  \end{center}
\end{table}

チーム編成を行い,学習者が活発に意見交換する創造的な学習活動を意図し,映像分析のグループワークを行なった.補完型グループについて,オブジェクトシーケン図による分析を行なった結果,継続的に相補的に意見交換が行われていることからグループ形成が有効であったと示唆した.



\section{関連研究まとめ}
先行研究では,人的リソースをバランスよく配分することによって全体的な効果は得られたが,個々の能力が創造活動にどのように反映されていたかなどは報告されていない.そこで本研究では,大きな創造性を生み出す潜在能力に着目し,研究を進めていく.あsだsだsだ

%--------------------------------------------------------------------
\chapter{グループワークの現状と個人の認識に関する調査}

グループ編成ツールの作成にあたり,  グループワークの現状を把握することを目的に調査を行なった.本章では,  調査の方法と目的,  結果,  考察について述べる.
\section{目的}

より多種多様なスキルを持つメンバーによるグループを構成するツールを制作するため,グループワークの現状を調査する.現状を把握することによって,問題を解決する方法を考える.

\section{調査方法}

本調査は,2018年6月27〜28日に,グループワークが多く行われている公立はこだて未来大学の18〜24歳の大学生,61名(男性49名, 女性12名)を対象に行われた.Googleフォームを用いて作成したアンケートを,回答者にメールにて送信し,回答してもらった.
\begin{figure}[h]
 \centering
   \includegraphics[width=80mm]{figures/groupwork1.png}
 \caption{Googleフォーム画面の一部}
 \label{fig:model}
\end{figure}


\subsection{質問内容}

1問目は,グループワークを通して新しいアイデアを考えたことがあるか.2問目は,新しいアイデアを考える際に良い結果が得られるのは個人ワーク,グループワーク,どちらでもないを選択,またその理由を記述してもらった.3問目は,初対面の人とグループワークをしたことがあるかを選択してもらった.4問目は,初対面の人とグループワークをする際に,その人の情報をあらかじめ知りたいと考えるかを選択,またその人の何を知りたいか記述してもらった.5問目は,グループで作業する際に自分と異なるスキルを持つ人と活動したいかを選択,またその理由を記述してもらった.6問目は,今までグループワークで困った経験はあるかを選択,またその理由を記述してもらった.


\section{調査結果}

アンケート調査の結果,全回答者61名のうち59名(96.7)がグループワークを通じて,何らかのアイデアを考えた経験があると回答した.グループワークをした経験がある59名のうち,50名(84.7)が新たなアイデアを考案する際に良い結果が得られるのはグループワーク,6名(10.2)がどちらでもない,3名(5.1)が個人ワークであると回答した.
(図\ref{graph0})
\begin{figure}[h]
 \centering
   \includegraphics[width=80mm]{figures/en1.png}
 \caption{支持をするのはどちらか}
 \label{graph0}
\end{figure}

\ グループワークを選んだ理由の記述を「様々な意見が得られる」「自分の視野が広がる」「意見を客観駅に見ることができる」「刺激をもらえる」「効率が上がる」の5つに分類した(図\ref{graph1}).
\begin{figure}[h]
 \centering
   \includegraphics[width=100mm]{figures/graph1.png}
 \caption{グループワークを支持する理由}
 \label{graph1}
\end{figure}

個人ワークを選ぶ理由として,「個人のほうが作業がはかどるから」などの意見などが得られた.どちらでもないを選ぶ理由として「両者に利点があると思うから」という意見などが得られた.
グループワークをした経験がある59名のうち,58名(98.3)が初対面の人とグループワークをした経験がある,1名(1.7)がないと回答した.
初対面の人とグループワークをしたことがあると回答した58名のうち,28名(48.3)が初対面の人の情報をあらかじめ知りたい,21名(36.2)がどちらでもない,9名(15.5)が知りたいと思わないと回答した 
初対面の人の情報をあらかじめ知りたいと回答した21名のうち,23名が初対面の人の得意分野,22名が性格,14名が趣味,9名が学校・職場,4名が生い立ち,5名がその他と回答した(図\ref{graph2}).

\begin{figure}[h]
 \centering
   \includegraphics[width=100mm]{figures/graph2.png}
 \caption{具体的に何を知りたいか}
 \label{graph2}
\end{figure}

グループワークをした経験がある59名のうち,57名(96.6)がグループワークを行う際に自分と異なるスキルを持った人と活動したい,2名(3.4)がどちらでもない,0名が活動したいと思わないと回答した.グループワークを行う際に自分と異なるスキルを持った人と活動したいと回答した57名のうち,活動したいと思う理由の記述データを「視野が広がる,成長できる」「作業の幅が広がる」「作業の効率が上がる」「その他」の4つに分類した.(図\ref{graph3})

\begin{figure}[h]
 \centering
   \includegraphics[width=100mm]{figures/graph3.png}
 \caption{スキルの違う人と作業をしたい理由}
 \label{graph3}
\end{figure}

グループワークをした経験がある59名のうち,50名(84.7)が今までグループワークをしていて困った経験がある,9名(15.3)が経験はないと回答した.グループワークで困った経験があると回答した50名の困った内容を「コミュニケーション」「意見の対立」「管理」「人間関係」「その他」の5つにカテゴリー分けて分類した.(図\ref{graph4})

\begin{figure}[h]
 \centering
   \includegraphics[width=100mm]{figures/graph4.png}
 \caption{グループワークで困ったこと}
 \label{graph4}
\end{figure}


\section{考察}

アンケート調査の結果から,グループワークを経験したことがある87\%が,知識の幅や視野が広がるなどの理由から,新たなアイデアを考案するプロセスでは良い結果が得られるのはグループワークであると認識していることが分かる.また,ほぼ同様の理由で96.6\%がグループワークで自分と異なるスキルを持った人と活動したいと考えていることと,82.1\%が初対面の人の知りたい情報として得意分野をあげている.これらのことから,回答者がグループワークに求めているものはアイデアの幅が広がることや,自分の持っていないスキルを持つ人と行動をすることで視野が広がることであると考えられる.\\
\ しかしグループワークを経験したことがある84.7\%が,知識,価値観の違いからコミュニケーションが進まなかった経験や意見の対立,グループの管理,人間関係などから,グループワークをしていて困った経験があると回答している.意見や価値観,知識の違いは視野を広げる可能性があるが,異なる背景を持つ人を理解する努力をしなければ,人と人が衝突し,逆にアイデアが決まらないことや,作業が進まなくなることが考えられる.
\ 初対面の人とグループワークをしたことがある52.7\%が,初対面の人とグループワークをする際に事前に情報を知りたいと思わない,どちらとも言えないと回答していることから,事前に初対面の人の情報を得ることに関しては消極的であることがわかる.\\
\ また事前に知りたい情報のうち性格と答えた人が78.6\%,趣味と答えた人が50\%であり,それらを高い割合で求めていることと,98.6\%が自分とスキルの異なる人と作業をしたいと考えていることから,知識やスキルに関しては自分と違うものを求めている

%--------------------------------------------------------------------
\chapter{システムの提案}
アンケート調査の結果とその考察から,大多数の人が自分と異なるスキルを持つ人と活動したいと考えていることが分かった.そこで,個人の得意分野にフォーカスをあてて,自分のスキルのデータを可視化し,グループで共有するWebアプリケーションの制作を行なった.

\section{提案コンセプト}
アンケート調査の結果からアイデアが広がる,自分の視野が広がるなどの理由からグループワークが支持されていることがわかった.
本研究の目的は個人のスキル、情報を可視化し、グループワークをする際に、より多種多様なスキルを持つメンバーによってグループを構成するためのツールを作成することである.そこで個人が持つスキルを平等に分配することでグループを作成し,作成したグループ内でグループメンバーのスキルを可視化するWebアプリケーションを提案する.可視化することで,グループメンバーへの理解やコミュニケーションの促進など,互いの理解が深まるの可能性が示唆される.またスキルの可視化はレーダーチャートを用いる.スキルの可視化以外にも自分の情報を入力し,メンバーに共有する機能も作成する.
\ 先行研究では教育者視点でのみのグループ配分をしており,グループワークをする側の視点に立っていない.そこで本システムでは可視化することでユーザー側の視点も導入することで,メンバーの相互理解を促進することが期待される.
\section{プロタイプ制作}

グループワークにおいて個人のスキルを平等に分配し,メンバーの情報を可視化するシステムのプロトタイプ制作を行った.
\begin{figure}[h]
 \centering
   \includegraphics[width=100mm]{figures/gamen.png}
 \caption{プロトタイプの一部}
 \label{gamen}
\end{figure}


\subsection{開発環境}
ここでは開発環境の説明をする.開発環境は以下の通りである.
\begin{table}[h]
\begin{center}
  \begin{tabular}{cll} \hline
    フロントエンド & Vue.js(JavaScript)  \tabularnewline
    バックエンド& Firebase\tabularnewline
    チャート &Chart.js \tabularnewline
    \hline
  \end{tabular}
  \caption{開発環境}
  \label{開発環境}
  \end{center}
\end{table}

開発言語はJavaScriptで、そのフレームワークであるVue.jsを使用した.Vue.jsを採用した理由は,Single Page Application(以下SPA)を作成できるからである.SPAのシステムを作成することでブラウザによるページを行わずにコンテンツの切り替えを行うことができ,ユーザー体験を大きく向上させることができる.
\ データベースとホスティング,ユーザー認証にはFirebaseを使用した.FirebaseはGoogle社が提供しているBaaS(Backend as a Service)である.Firebaseを採用した理由は,バックエンドの実装を容易にすることができる点と高速な非同期通信をすることができる点である.Firebaseを用いることで,より高速なデータ通信を行うことができ,ユーザー体験を向上させることができる.
\ またスキルのチャート表示にはJavaScriptのライブラリである,Chart.jsを用いた.Chart.jsにデータを与えると,容易にレーダーチャートを表示することができる.
\subsection{画面構造}
システムの画面遷移図を図〇〇で表す.システムはディスプレイの大きさに合わせてレイアウトを変えている.
\ ユーザーはログイン画面でログインを行う,ログインの実装にはFirebaseのユーザー認証機能であるFirebase Authenticationを利用している.Firebase Authenticationを用いることで高速でユーザー認証を行うことができる.ユーザー認証はGoogleアカウントを用いて行う.ログインしたのちにユーザーの情報を記載しているホーム画面に遷移する.ホーム画面では自分のスキルを表示したレーダーチャートと基本情報が記載されている.また画面右上のメニューボタンを押すことでメニューが画面右側に表示される.メニューからグループ生成画面,グループ画面,設定画面,テスト画面に遷移することができる.グループ生成画面では次項で説明する.グループ画面では作成されたグループの情報が記載されている.具体的には,所属しているメンバー,メンバー全員のスキルを表示したチャート,チャットが記載されている.所属したメンバーはプロフィール画像で表示しており,プロフィール画像を押すことでメンバーのプロフィール画面に遷移することができる.メンバーのプロフィール画面はホーム画面の構造と同じである.スキルを表示したチャートはメンバー全員のスキルをそのまま表示したチャート,メンバー間の最大値を表示したチャート,メンバー間の平均値を表示したチャートの3種類ある.またチャットはメンバーが投稿したメッセージを非同期で表示することができる.

\subsection{グループ自動生成機能}




%--------------------------------------------------------------------
\chapter{実験}

\section{評価結果}

Java言語との比較では,惨敗であり,FUNは2倍の
記述量を必要とした.しかし,これは,Javaのもつ
パッケージIKURAが非常に強力であるためで,
同一機能をもつライブラリを用意することにより,
FUNにも同様の能力を持たせることができることが判明した.

\section{評価結果}

Java言語との比較では,惨敗であり,FUNは2倍の
記述量を必要とした.しかし,これは,Javaのもつ
パッケージIKURAが非常に強力であるためで,
同一機能をもつライブラリを用意することにより,
FUNにも同様の能力を持たせることができることが判明した.


%--------------------------------------------------------------------
\chapter{結論と今後の展開}

\section{まとめ}

Java言語との比較では,惨敗であり,FUNは2倍の
記述量を必要とした.しかし,これは,Javaのもつ
パッケージIKURAが非常に強力であるためで,
同一機能をもつライブラリを用意することにより,
FUNにも同様の能力を持たせることができることが判明した.

Java言語との比較では,惨敗であり,FUNは2倍の
記述量を必要とした.しかし,これは,Javaのもつ
パッケージIKURAが非常に強力であるためで,
同一機能をもつライブラリを用意することにより,
FUNにも同様の能力を持たせることができることが判明した.

Java言語との比較では,惨敗であり,FUNは2倍の
記述量を必要とした.しかし,これは,Javaのもつ
パッケージIKURAが非常に強力であるためで,
同一機能をもつライブラリを用意することにより,
FUNにも同様の能力を持たせることができることが判明した.

\section{今後の方針}

Java言語との比較では,惨敗であり,FUNは2倍の
記述量を必要とした.しかし,これは,Javaのもつ
パッケージIKURAが非常に強力であるためで,
同一機能をもつライブラリを用意することにより,
FUNにも同様の能力を持たせることができることが判明した.


%--------------------------------------------------------------------
\chapter*{謝辞}

本研究において、長期にわたる評価実験に協力いただきました、株式会社○○の△△△△様に感謝いたします.


%--------------------------------------------------------------------
% 参考文献
\begin{thebibliography}{9}
 \bibitem {A1} 上平崇仁:協調的デザイン学習における人間中心設 計プロセスの適用, 専修大学情報科学研究所所報, 2011
\bibitem {A2}石井成郎, 三輪和久:創造活動における心的操作と外敵操作のインタラクション, 認知科学, Vol.10, No.4, pp.469-485, 2003.
\bibitem {A3}Kang,N : Proposal and Evaluation of Design
Support Tools for Logical Collaborative Design Process, Archives of design research 2015, vol28, No.4, pp.63-75, 2015.
\bibitem {A4}亀田達也 : 合議の知を求めて—グループの意志決定,共立出版,1997.

\bibitem {A5}相島雅樹, 塚原康仁, 植木淳郎, 杉浦一徳:DTMPメソッドを用いたグループ編成システムの提案, 研究報告情報システムと社会環境, Vol.2011-IS-115, No.2, pp.1-8, 2011.

\bibitem {A6}橋本弘明, 桑原徹, 秋玉梅, 石川達也, 山下公太郎, 古宮誠一:ソフトウェア開発グループ演習のためのチーム編成の最適化支援, メディア教育研究, Vol.3, No.2, pp.61-69, 2007. 

\bibitem {A7}井上久祥,埴生加奈子:学習者の思考特性に着目したグループ形成支援の方法—協調作業を有効にするグループ形成支援システムのための基礎的研究—,情報処理学会研究報告,2004-GN-53,pp.19-24,2004

\bibitem {A8}小林恵智:プロジェクトリーダーのための [入門] チームマネジメント,PHP研究所,2001

\bibitem {A9}北村清一:FFS理論を活用した成果を生み出すチーム編成と実践的な作業割り振りの提案,プロジェクトマネジメント学会研究発表大会予稿集 2013.Autumn(0), pp.115-120, 2013


\end{thebibliography}


% 以降,付録(付属資料)であることを示す
\appendix

%--------------------------------------------------------------------
\chapter*{付録その1} % \chapter{}を使うと「付録A ***」となる

付録その1(プログラムのソースリストなど)を必要があれば載せる

%--------------------------------------------------------------------
\chapter*{付録その2}

付録その2(関連資料など)を必要があれば載せる

%--------------------------------------------------------------------
% 図一覧
\listoffigures

%--------------------------------------------------------------------
% 表一覧
\listoftables

\end{document}
