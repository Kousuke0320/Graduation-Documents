% 公立はこだて未来大学 卒業論文 テンプレート ver1.50
% (c) Junichi Akita (akita@fun.ac.jp), 2003.10.31
% update by N.T.,  2004.11.10
%
\documentclass{funthesis}
%\documentclass[english]{funthesis} % use [english] option for English style

\usepackage{graphicx} % 図(EPS形式)を本文中で読み込む場合はこれを宣言

% この部分に,タイトル・氏名などを書く.
% タイトルなどの定義の始まり
\jtitle{デザインプロセスにおけるグループワーク編成支援ツールの提案\\
--- サブタイトル ---
}  % 論文の和文タイトル
%
\etitle{Suggestion of group work support tool in the design process\\
--- English Subtitle ---
}% 論文の英文タイトル
%
\htitle{Short Title in English}   % ヘッダー用の論文の短縮英文タイトル
%     必ず1行に収まるように英文タイトルを短縮¥¥¥¥¥¥¥¥¥¥¥¥¥¥¥¥

%
\jauthor{田中 康介}     % 氏名(日本語)
\eauthor{Kosuke Tanaka}   % 氏名(英語)
\jaffiliciation{情報アーキテクチャ学科} % 所属学科名(日本語)
\eaffiliciation{Department of Media Architecture} % 所属学科名(英語)
\studentnumber{1015013}   % 学籍番号
\jadvisor{姜 南圭}    % 正指導教員名(日本語)
\eadvisor{Prof. Kang}  % 正指導教員名(英語)
\jdate{2019年1月29日}    % 論文提出日   (日本語)
\edate{January 29, 2019}     % 論文提出年月 (英語)
% タイトルなどの定義の終わり

\begin{document}

%--------------------------------------------------------------------
\maketitle       % タイトルページを作成

%--------------------------------------------------------------------
% 英文概要(250語程度)
\begin{eabstract}
Currently, problems that designers have to deal with are considering the relationship with the users and the society. Therefore, it becomes difficult to find the best answer with the sensitivity and creativity of individual designers.As a result, it is required to organize teams by members with various skills and to conduct creative activities systematically.In the previous studies, results are gotten by distributing human resources in a well-balanced manner.But, it has not been reported what kind of influence the member composition has on creative activities.Consequently, we conducted a survey about group work in creative activities.As a result, most people chose group work rather than personal work for different reasons such as expanding their own vision.In this research, we suggest a web application, named ‘Skill Pentagon’ for organizing more various groups by visualizing the user's skill.\end{eabstract}

% 英文キーワード(5個程度をコンマ(,)で区切って羅列する)
\begin{ekeyword}
Design Process, Group work, Web application, Creativity
\end{ekeyword}

%--------------------------------------------------------------------
% 和文概要(400字程度)
\begin{jabstract}
現在デザイナーが対処すべき問題は,ユーザや社会との関係性をより丹念に考慮することが 求められるようになり,デザイナー個人の感性や創造性だけでは最適な答えを出すことは難しくなっている.このことから,多種多様なスキルを持つメンバーによってチームを組み,組織的に創造 活動をすることが求められている.先行研究では,人的リソースをバランスよく配分することで全体的な成果を挙げたことが報告されている.しかし,個人の創造活動がメンバー構成によってどのように変化したかは報告されていない.そのため,公立はこだて未来大学にて創造活動におけるグ ループワークの現状についてのアンケート調査を行った.その結果,視野が広がるなどの理由から,自分と異なるスキルを持つ人とグループワークを支持する人が大多数であるということが分かった. 本稿ではユーザのスキルを可視化することにより,より多種多様なグループ編成を行うための Web アプリケーションの提案を行う.
\end{jabstract}

% 和文キーワード(5個程度をコンマ(,)で区切って羅列する)
\begin{jkeyword}
デザインプロセス, グループワーク,ウェブアプリケーション, 創造性
\end{jkeyword}

%--------------------------------------------------------------------
\tableofcontents % 目次を作成


% 本文のはじまり
%--------------------------------------------------------------------
\chapter{序論} % 章のタイトル
%\chapter{Introduction} % sample of English style

本章では、本研究における背景と、研究目的について述べる.

% \includegraphics[width=??cm]{hoge.eps} % 図(EPS形式)を読み込む場合

\section{背景} % sectionのタイトル

% 以下に背景,関連する環境,状況,技術に関する概要を記述.

現在デザイナーが対処すべき問題は,ユーザや社会との関係性をより丹念に考慮することが求められるようになり,デザイナー個人の感性や創造性だけでは最適な答えを出すことは難しくなっている.そこで個人の創造性を超えて,多角的な方面からより創造的な解を生み出すために多様な専門性を持つチームによる組織的なデザイン行為のあり方について議論されることが増えてきた.\cite{A1} 石井らは創造性という観点から一般的な創造活動の認知モデルであるジェネプロアモデルの枠組みを適応し,問題解決においてアイデアを検討する段階に限定したうえで,「独立して考える場合と比較して二人で話し合うという協同には効果がある」と示した\cite{A2}. また,Kangは異なる学問や経験,文化を持っている人が集まりグループワークを行うことは,大きな創造性を生み出す潜在能力を持っている\cite{A3}としている.\\
\ これらのことから,自分と異なるスキルや視点を持つ人と共に創造的な作業を行うことにより,自分だけの潜在能力では考えることができなかった成果をあげることがあると考えられる.\\
\ しかし、グループワークでは, グループを形成するメンバーによっては、手を抜く学生が生じ、一人に過剰な負荷がかかるなどの問題も指摘されていることもあり\cite{A4}様々なグループ形成方法が議論されているのが現状である.

\section{研究目的}
本研究では創造性とグループワークに着目し,  個人のスキル,情報を可視化し,グループワークをする際により多種多様なスキルを持つメンバーによってグループを構成するためのツールの提案を行う.
%--------------------------------------------------------------------
\chapter{関連研究}

\section{創造性とグループワーク}

\subsection{Smalltalk-80} % subsectionのタイトル

Smalltalk-80は1982年ごろ,当時ゼロックスにいた...

%\subsubsection{必要があれば} % subsubsectionのタイトル
% ※ subsubsectionはあまり使わないほうがよい

\subsection{Java 3D}

Javaはオブジェクト指向言語で,そこで3D グラフィックスを扱うための..

\section{グループ編成システム}

\subsection{DTMPメソッドを用いたグループ編成システム}

相島・塚原・植木.杉浦\cite{A5}は学習プロセス方法論であるDTMPメソッドをユーザのふるまいに当てはめ,人的リソースを配分する手法を提案し,ウェブアプリケーションを実装,グループ学習において検証を行った.DTMPメソッドとはDTMPの各系統の能力について以下の図のように定義したものである表\ref{DTMP}.
\begin{table}[h]
\begin{center}
  \begin{tabular}{cll} \hline
    D & Design & 創造的な本質を理解し,創造活動を実践できる能力\tabularnewline
    T & Technology & 
    創造的活動を支えるデジタルメディア関連の技術を理解・活用できる能力\tabularnewline
    M &Management & 
    創造的デザインプロセスを管理できる能力\tabularnewline
    P &Polisy &
    創造的活動の成果を戦略的に活用でき,創造的プロセスを取り巻く政策を理解できる能力\tabularnewline
    \hline
  \end{tabular}
  \caption{DTMPメソッドの定義}
  \label{DTMP}
  \end{center}
\end{table}

\ ふるまいはユーザの潜在的な特性を反映しており,DTMP系統の特性を対応づけることにより,個人が持つDTMP特性を抽出できる.\\
\ 検証を行った結果,人的リソースのバランス良い配分という観点から効果を認めることができたが,DTMPメソッドの各特性が持つ能力や傾向を確認することはできなかった.

\subsection{ソフトウェア開発グループ演習のためのチーム最適化支援}
橋浦・桑原・秋・石川・山下・古宮\cite{A6}はソフトウェア開発におけるグループ内の役割をリーダー,設計担当,コーディング担当,品質管理担当に分け,それぞれに必要な適正を成績から代用し,遺伝的アルゴリズムを使用してチーム編成を行う手法を提案した.ソフトウェア開発グループ演習において検証を行った結果,学習者の満足度が上がったことと,チーム間の成果物のバラつきが軽減されたことから,ソフトウェア開発演習のための最適なチーム編成が実現できていることを確認した.
DirectX はマイクロソフトのWindows上の.....

\section{関連研究まとめ}


%--------------------------------------------------------------------
\chapter{プログラミング言語FUN}

この章では,提案する理論,仮説,モデル,アルゴリズム,
方法論,実装のなどの説明を行う.

\section{提案する言語FUNの特徴}

この言語の特徴は,..であり,...という従来にない長所をもつ.

\section{言語仕様}

言語仕様は以下の通り.


\section{実装方法}

この言語は,C言語を用いて記述されている.ソースコードは20に分かれ,
コードの大きさは約3000行となった.

\subsection{開発環境}

この言語は,C言語を用いて記述されている.ソースコードは20に分かれ,
コードの大きさは約3000行となった.

\subsection{OSに対する依存性}

この言語は,C言語を用いて記述されている.ソースコードは20に分かれ,
コードの大きさは約3000行となった.


%--------------------------------------------------------------------
\chapter{実験と評価}

\section{保守性に関する評価}

ここでは,FUNを用いて記述した場合と
それ以外の言語で書いた場合の比較を行なう.

\subsection{Fortranとの比較}

同一のゲームをFortranとFUNで記述してみた.

\subsubsection{スーパーマリオブラザーズ}

一見,このプログラムはFortran向きと考えられるが,
FUNのTAKOIKAライブラリを用いて記述すると,
非常にコンパクトになる.

\subsubsection{パックマン}

このプログラムはどちらの言語にとっても,
有利な要素はない,このことを反映して.

\subsection{Javaとの比較}

Java言語との比較では,惨敗であり,FUNは2倍の
記述量を必要とした.しかし,これは,Javaのもつ
パッケージIKURAが非常に強力であるためで,
同一機能をもつライブラリを用意することにより,
FUNにも同様の能力を持たせることができることが判明した.

\section{実行速度}

\subsection{Fortranとの比較}

Java言語との比較では,惨敗であり,FUNは2倍の
記述量を必要とした.しかし,これは,Javaのもつ
パッケージIKURAが非常に強力であるためで,
同一機能をもつライブラリを用意することにより,
FUNにも同様の能力を持たせることができることが判明した.

\subsection{Javaとの比較}

Java言語との比較では,惨敗であり,FUNは2倍の
記述量を必要とした.しかし,これは,Javaのもつ
パッケージIKURAが非常に強力であるためで,
同一機能をもつライブラリを用意することにより,
FUNにも同様の能力を持たせることができることが判明した.

\section{利用者によるアンケート}

\subsection{初心者}

Java言語との比較では,惨敗であり,FUNは2倍の
記述量を必要とした.しかし,これは,Javaのもつ
パッケージIKURAが非常に強力であるためで,
同一機能をもつライブラリを用意することにより,
FUNにも同様の能力を持たせることができることが判明した.

\subsection{上級者}

Java言語との比較では,惨敗であり,FUNは2倍の
記述量を必要とした.しかし,これは,Javaのもつ
パッケージIKURAが非常に強力であるためで,
同一機能をもつライブラリを用意することにより,
FUNにも同様の能力を持たせることができることが判明した.


%--------------------------------------------------------------------
\chapter{考察}

\section{評価結果}

Java言語との比較では,惨敗であり,FUNは2倍の
記述量を必要とした.しかし,これは,Javaのもつ
パッケージIKURAが非常に強力であるためで,
同一機能をもつライブラリを用意することにより,
FUNにも同様の能力を持たせることができることが判明した.

\section{評価結果}

Java言語との比較では,惨敗であり,FUNは2倍の
記述量を必要とした.しかし,これは,Javaのもつ
パッケージIKURAが非常に強力であるためで,
同一機能をもつライブラリを用意することにより,
FUNにも同様の能力を持たせることができることが判明した.


%--------------------------------------------------------------------
\chapter{結論と今後の展開}

\section{まとめ}

Java言語との比較では,惨敗であり,FUNは2倍の
記述量を必要とした.しかし,これは,Javaのもつ
パッケージIKURAが非常に強力であるためで,
同一機能をもつライブラリを用意することにより,
FUNにも同様の能力を持たせることができることが判明した.

Java言語との比較では,惨敗であり,FUNは2倍の
記述量を必要とした.しかし,これは,Javaのもつ
パッケージIKURAが非常に強力であるためで,
同一機能をもつライブラリを用意することにより,
FUNにも同様の能力を持たせることができることが判明した.

Java言語との比較では,惨敗であり,FUNは2倍の
記述量を必要とした.しかし,これは,Javaのもつ
パッケージIKURAが非常に強力であるためで,
同一機能をもつライブラリを用意することにより,
FUNにも同様の能力を持たせることができることが判明した.

\section{今後の方針}

Java言語との比較では,惨敗であり,FUNは2倍の
記述量を必要とした.しかし,これは,Javaのもつ
パッケージIKURAが非常に強力であるためで,
同一機能をもつライブラリを用意することにより,
FUNにも同様の能力を持たせることができることが判明した.


%--------------------------------------------------------------------
\chapter*{謝辞}

本研究において、長期にわたる評価実験に協力いただきました、株式会社○○の△△△△様に感謝いたします.


%--------------------------------------------------------------------
% 参考文献
\begin{thebibliography}{9}
 \bibitem {A1} 上平崇仁:協調的デザイン学習における人間中心設 計プロセスの適用, 専修大学情報科学研究所所報, 2011
\bibitem {A2}石井成郎, 三輪和久:創造活動における心的操作と外敵操作のインタラクション, 認知科学, Vol.10, No.4, pp.469-485, 2003.
\bibitem {A3}Kang,N : Proposal and Evaluation of Design
Support Tools for Logical Collaborative Design Process, Archives of design research 2015, vol28, No.4, pp.63-75, 2015.
\bibitem {A4}亀田達也 : 合議の知を求めて—グループの意志決定,共立出版,1997.

\bibitem {A5}相島雅樹, 塚原康仁, 植木淳郎, 杉浦一徳:DTMPメソッドを用いたグループ編成システムの提案, 研究報告情報システムと社会環境, Vol.2011-IS-115, No.2, pp.1-8, 2011.

\bibitem {A6}橋本弘明, 桑原徹, 秋玉梅, 石川達也, 山下公太郎, 古宮誠一:ソフトウェア開発グループ演習のためのチーム編成の最適化支援, メディア教育研究, Vol.3, No.2, pp.61-69, 2007. 
 
\end{thebibliography}


% 以降,付録(付属資料)であることを示す
\appendix

%--------------------------------------------------------------------
\chapter*{付録その1} % \chapter{}を使うと「付録A ***」となる

付録その1(プログラムのソースリストなど)を必要があれば載せる

%--------------------------------------------------------------------
\chapter*{付録その2}

付録その2(関連資料など)を必要があれば載せる

%--------------------------------------------------------------------
% 図一覧
\listoffigures

%--------------------------------------------------------------------
% 表一覧
\listoftables

\end{document}
